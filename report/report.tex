\documentclass{article}

\usepackage[utf8]{inputenc}
\usepackage[brazil]{babel}
\usepackage[a4paper, left=3cm, right=2cm, top=3cm, bottom=2cm]{geometry}
\usepackage{indentfirst}
\usepackage[]{graphicx}
\usepackage{amsmath}
\usepackage{float}
\usepackage{lipsum}
\usepackage{xcolor}
\usepackage{fancyvrb}
\usepackage{verbatimbox}
\usepackage{tikz}
\usepackage{float}

\catcode`>=\active %
\catcode`<=\active %
\def\openesc{\color{red}}
\def\closeesc{\color{black}}
\def\vbdelim{\catcode`<=\active\catcode`>=\active%
\def<{\openesc}
\def>{\closeesc}}
\catcode`>=12 %
\catcode`<=12 %

\newcommand{\assignment}{Trabalho 1}
\newcommand{\duedate}{23 de Abril}

\title{
    Relatório do \assignment \\
    % Subtítulo
    Pipeline de Processamento de Dados - Simulação de Rodovias
}
\author{
    Breno Marques Azevedo \\
    Bruno Pereira Fornaro \\
    Luis Fernando Laguardia \\
    Vanessa Berwanger Wille \\
    Vinicius Hedler
}
\date{\today}

\begin{document}
    \noindent
    Fundação Getulio Vargas\hfill\\
    Computação Escalável\hfill\textbf{\assignment}\\
    Prof.\ Thiago Pinheiro de Araújo\hfill\textbf{Entrega:} \duedate\\
    \smallskip\hrule\bigskip

    {\let\newpage\relax\maketitle}
    \maketitle

    \section{Introdução}
    Neste trabalho iremos implementar um pipeline de processamento de dados
    para um sistema de monitoramento de rodovias, seguindo o modelo ETL 
    (Extract, Transform, Load) e utilizando os mecanismos apresentados
    em aula para executar de forma concorrente e paralela.

    \section{Modelagem e Distribuição de Tarefas}
    Como o trabalho é dividido em segmentos bem definidos, atribuímos tarefas
    à cada integrante do grupo.

    \begin{figure}[H]
        \centering
        \includegraphics*[width=12cm]{figs/modelagem.jpg}
        \caption{Distribuição de tarefas.}
        \label{fig:modelagem}
    \end{figure}

    Inicialmente, definimos bem como seria o funcionamento do Mock, a fim de projetar
    como o que esperar durante o processamento dos dados. Isso nos ajudou a 
    compartimentalizar o trabalho de forma relativamente independente. Contudo, ainda
    mantivemos uma comunicação constante entre os membros do grupo. Realizamos reuniões
    periódicas para discutir como implementar cada parte do trabalho, falar sobre o 
    andamento do projeto e debater problemas que surgiam durante o desenvolvimento.

    \subsection*{Mock}
    Primeiramente, uma vez que implementar um sistema de monitoramento de rodovias
    não é o objetivo do trabalho, criamos um Mock que imita o comportamento que um sistema desse tipo teria.
    O Mock simula uma \textbf{rodovia} e os \textbf{carros} que passam por ela, bem como as \textbf{colisões} 
    que podem ocorrer. Por conta disso, decidimos que o Mock teria cada uma dessas três classes.
    
    Cada carro possui uma série de atributos, tais como placa, modelo, velocidades máxima e mínima, 
    acelerações máxima e mínima, uma probabilidade de colisões e uma probabilidade de trocar de faixa.
    Dessa forma, é possível que o carro siga em frente, troque de faixa ou cause algum tipo de colisão.

    Por sua vez, cada rodovia é composta por um comprimento, número de faixas em cada sentido,
    um limite de velocidade e uma lista que armazena uma lista de carros e outra lista de colisões.

    Por fim, as colisões são representadas por uma lista de carros que colidiram e as respectivas
    contagens desses carros que colidiram.

    Como sugerido na especificação do trabalho, a simulação é atualizada por ciclos. A cada ciclo,
    os carros se movem conforme possível, dados os fatores como velocidade, aceleração, colisões,
    probabilidade de trocar de faixa, etc. Além disso, quando há uma colisão, o Mock realiza uma
    contagem regressiva para removê-la, que é atualizada a cada ciclo.

    Por fins de simplicidade, decidimos que a rodovia funcionaria como uma grande matriz, onde cada
    célula representa um segmento da rodovia, podendo ou não conter um carro ou colisão. Dessa forma,
    definir a posição do carro se torna uma tarefa mais simples, pois definir uma simulação mais complexa
    e precisa de como os carros se movem e colidem está fora do escopo do trabalho. O que realmente nos
    importa é o processamento dos dados gerados pelo Mock.

    \subsection*{ETL}
    \lipsum[1]

    \subsection*{Dashboard}
    \lipsum[2]

    \section{Mock}
    Como mencionado acima, o Mock é uma simulação de um sistema de monitoramento de rodovias.
    Esse sistema é como um sensor em uma rodovia, relativamente parecido com um radar, mas ele
    grava apenas a rodovia em que se encontra, a placa dos carros que passam por ela e a posição
    desses carros.

    Durante a execução do Mock, é possível ver a simulação da rodovia, o movimento dos carros a cada
    ciclo e as colisões que ocorrem. Segue abaixo um exemplo:

    \begin{verbnobox}[\vbdelim]
2                                                     <9>                                                                                                           2                                                                                                                                        
 2                                                        <9>                 
                               2                                            
------------------------------------------------------------------------------------
                                                 -4                         
                                                 <8>                        -3         
             -2
    \end{verbnobox}
    
    Acima, temos uma simulação de uma rodovia com 3 faixas em cada sentido. Os números representam
    as acelerações. O sinal negativo representa que esses carros estão indo no sentido oposto ao
    da faixa de cima. Por sua vez, os números em vermelho são as contagens regressivas das colisões,
    que são atualizadas a cada ciclo.

    Além disso, a simulação também gera um arquivo de saída (\textit{output.txt}) que contém o nome
    da rodovia observada,as placas dos carros que passam por ela e suas respectivas posições a cada
    ciclo. A posição do veículo é representada por um par ordenado com o número da via e a distância
    percorrida. Segue abaixo um exemplo de arquivo de output, referente à simulação acima:

    \begin{verbatim}
> BR-286
DQF9D30 000,000
TWB3V34 001,001
EOI9E67 002,010
DWQ5P39 003,017
JYX8N25 004,007
IFK9V68 005,034
    \end{verbatim}

    Por fim, o arquivo de saída será usado durante o processamento dos dados. É esperado que várias
    simulações sejam realizadas, gerando um volume dados considerável. Que permitirá análises interessantes
    depois de processado.

    \section{ETL}

    \section{Dashboard}

    \section{Problemas e Soluções}
    Um problema que enfrentamos logo nos momentos iniciais do projeto, foi a modelagem do Mock. Principalmente
    no que diz respeito à representação da rodovia e das colisões. A nossa solução, como mencionado durante a 
    a seção de modelagem, foi definir a rodovia como uma grande matriz e representar as células como possíveis
    carros e colisões. Isso foi importante, pois uma das formas que imaginamos era definir um comprimento para o 
    carro definir um tipo de \textit{hitbox} para ele. Entretanto, isso estaria muito longe do escopo do projeto
    e essa simplicidade nos permitiu agilizar o desenvolvimento do simulador.

    \section{Conclusão}

\end{document}
