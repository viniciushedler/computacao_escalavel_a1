\documentclass{article}

\usepackage[utf8]{inputenc}
\usepackage[brazil]{babel}
\usepackage[a4paper, left=3cm, right=2cm, top=3cm, bottom=2cm]{geometry}
\usepackage{indentfirst}
\usepackage[]{graphicx}
\usepackage{amsmath}
\usepackage{float}
\usepackage{lipsum}
\usepackage{xcolor}
\usepackage{fancyvrb}
\usepackage{verbatimbox}

\catcode`>=\active %
\catcode`<=\active %
\def\openesc{\color{red}}
\def\closeesc{\color{black}}
\def\vbdelim{\catcode`<=\active\catcode`>=\active%
\def<{\openesc}
\def>{\closeesc}}
\catcode`>=12 %
\catcode`<=12 %

\newcommand{\assignment}{Trabalho 1}
\newcommand{\duedate}{23 de Abril}

\title{
    Relatório do \assignment \\
    % Subtítulo
    Pipeline de Processamento de Dados - Simulação de Rodovias
}
\author{
    Breno Marques Azevedo \\
    Bruno Pereira Fornaro \\
    Luis Fernando Laguardia \\
    Vinicius Hedler \\
    Vanessa Wille 
}
\date{\today}

\begin{document}
    \noindent
    Fundação Getulio Vargas\hfill\\
    Computação Escalável\hfill\textbf{\assignment}\\
    Prof.\ Thiago Pinheiro de Araújo\hfill\textbf{Entrega:} \duedate\\
    \smallskip\hrule\bigskip

    {\let\newpage\relax\maketitle}
    \maketitle

    \section{Introdução}
    Neste trabalho iremos implementar um pipeline de processamento de dados
    para um sistema de monitoramento de rodovias, seguindo o modelo ETL 
    (Extract, Transform, Load) e utilizando os mecanismos apresentados
    em aula para executar de forma concorrente e paralela.

    \section{Modelagem}
    Aqui podemos falar da forma que distribuímos as tarefas(?).

    \subsection*{Mock}
    Primeiramente, uma vez que implementar um sistema de monitoramento de rodovias
    não é o objetivo do trabalho, criamos um Mock que imita o comportamento que um sistema desse tipo teria.
    O Mock simula uma rodovia e os carros que passam por ela, bem como as colisões que podem ocorrer.
    Por conta disso, decidimos que o Mock teria cada uma dessas três classes.
    
    Cada carro possui uma série de atributos, tais como placa, modelo, velocidades máxima e mínima, 
    acelerações máxima e mínima, uma probabilidade de colisões e uma probabilidade de trocar de faixa.
    Dessa forma, é possível que o carro siga em frente, troque de faixa ou cause algum tipo de colisão.

    Por sua vez, cada rodovia é composta por um comprimento, número de faixas em ambos os sentidos, 
    uma lista que armazena uma lista de carros e outra lista de colisões, um limite de velocidade 
    e uma contagem regressiva para as colisões.

    Por fim, as colisões são representadas por uma lista de carros que colidiram e as respectivas
    contagens desses carros que colidiram.

    \subsection*{ETL}
    \lipsum[1]

    \subsection*{Dashboard}
    \lipsum[2]

    \section{Mock}
    Como mencionado acima, o Mock é uma simulação de um sistema de monitoramento de rodovias.
    Ele é composto pela rodovia, os carros e as colisões. É possível visualizar a simulação
    ocorrendo a cada ciclo. Além disso, ele gera como saída um arquivo com o nome da rodovia
    observada, os carros que passam por ela e suas respectivas posições - o número da via e 
    distância percorrida.
    
    Segue um exemplo de simulação do Mock:

    \begin{verbnobox}[\vbdelim]
2                                                     <9>                                                                                                           2                                                                                                                                        
 2                                                        <9>                 
                               2                                            
------------------------------------------------------------------------------------
                                                 -4                         
                                                 <8>                        -3         
             -2
    \end{verbnobox}
    
    Acima, temos uma simulação de uma rodovia com 3 faixas em cada sentido. Os números representam
    as acelerações. O sinal negativo representa que esses carros estão indo no sentido oposto ao
    da faixa de cima. Já os números em vermelho são as contagens regressivas das colisões, que são
    atualizadas a cada ciclo.

    Além disso, a simulação também gera um arquivo de output que contém o nome da rodovia observada,
    as placas dos carros que passam por ela e suas respectivas posições a cada ciclo. A posição do
    veículo é representada por um par ordenado com o número da via e a distância percorrida.
    Segue abaixo um exemplo de arquivo de output, referente à simuação acima:

    \begin{verbatim}
> BR-286
DQF9D30 000,000
TWB3V34 001,001
EOI9E67 002,010
DWQ5P39 003,017
JYX8N25 004,007
IFK9V68 005,034
    \end{verbatim}

    O output em questão será usado como entrada para o processo de ETL.

    \section{ETL}

    \section{Dashboard}

    \section{Problemas e Soluções}

    \section{Conclusão}

\end{document}
